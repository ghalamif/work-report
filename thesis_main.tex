% Document class for LME theses: lmedoc
% %LANGUAGE
% %CONFIG
%    The option "german" uses german.sty
%    For english papers, use the "english" option
% Possible types of theses:
% bt - Bachelor's thesis
% mt - Master's thesis
% diss - Dissertation
% sa - Student thesis
% fp - Forschungspraktikum
% pa - Projektarbeit
%\documentclass[german,bt]{lmedoc/lmedoc}
\documentclass[english,fp]{lmedoc/lmedoc}


%%%%%%%%%%%%%%%%%%
% pdflatex and lualatex are supported
% ++ "Umlaut" support
%    The package "inputenc" can be used to write Umlaute or the german double s
%    directly. You need to use the correct encoding, e.g. latin1.
\usepackage{iftex}
\ifPDFTeX
  \usepackage[utf8]{inputenc}
  \usepackage[T1]{fontenc}
  \usepackage{lmodern}
\else
  \ifXeTeX
     \usepackage{fontspec}
  \else 
     \usepackage{luatextra}
  \fi
\fi

%Load additional packages
\usepackage{lmedoc/packages}
% put in this file your abbreviations
% add your abbreviations here
\newabbreviation{svm}{SVM}{Support Vector Machine}

%%%%%%%%%%%%%%%%%%%%%%%%%%%%

% supress messages because of underful hbox, this is not a problem
\hbadness=10000



% some useful commands
\makeatletter % let's define a single dot
\DeclareRobustCommand\onedot{\futurelet\@let@token\@onedot}
\newcommand{\@onedot}{\ifx\@let@token.\else.\null\fi\xspace}
\makeatother



% Sets the bib file

\addbibresource{literature.bib}

% When writing a large document, it is sometimes useful to work on selected sections of the document.
% Use this command to only build the document partially. Speeds up the developement cycle.
% For the final product, this has to be commented out.
%\includeonly{introduction,appendix,foo,bar}


\pagenumbering{roman}

\begin{document}
\clearpage
% %CONFIG
% This is for students' theses
  \begin{deckblatt}
    \Titel{Titel} % Title
    \Name{Name} % Last name
    \Vorname{Vorname} % Given name
    \Geburtsort{Geburtsort} % Place of birth
    \Geburtsdatum{Geburtsdatum} % Date of birth
    \Betreuer{Betreuer} % Advisor
    \Start{Start} % Start of thesis
    \Ende{Ende} % End of thesis
    %\ZweitInstitut{ZweitInstitut} % Cooperation partner
  \end{deckblatt}


\cleardoublepage
% Only necessary for thesis

Ich versichere, dass ich die Arbeit ohne fremde Hilfe und ohne Benutzung
anderer als der angegebenen Quellen angefertigt habe und dass die Arbeit
in gleicher oder "ahnlicher Form noch keiner anderen Pr"ufungsbeh"orde
vorgelegen hat und von dieser als Teil einer Pr"ufungsleistung
angenommen wurde. Alle Ausf"uhrungen, die w"ortlich oder sinngem"a"s
"ubernommen wurden, sind als solche gekennzeichnet.
\\


\cleardoublepage


\begin{center}
\bfseries
% Abstract in German
{\selectlanguage{german}"Ubersicht}
\normalfont
\end{center}


\vspace{5.0cm}

\begin{center}
\bfseries
% Abstract in English
{\selectlanguage{english}Abstract}
\normalfont
\end{center}


\cleardoublepage

\tableofcontents

\cleardoublepage \pagenumbering{arabic}

\chapter{Introduction}

% let's show you how \cite and \gls for abbreviations works
Example paragraph: 

Niemann~\cite{Niemann83KVM} wrote a nice book. There is no \glspl{svm} defined,
but maybe he said something about Bayesian classifiers, maybe here~\cite[p.34]{Niemann83KVM}.
But I like \glspl{svm}, so give me an \gls{svm}. 

Useful reads:

Checkout the subcaption package how to do multiple figures/tables. Make sure
that you use vector graphics - no blurry png - for graphs or similar, eg use
tikz/inkscape. \cref{fig:ex} is an example of a figure using the package tikz. Make also sure that your plots are readable and have axis
captions, eg use pgfplots.

How to create good looking tables with the booktabs package
e.g. a table should like \cref{tab:ex}.

\begin{table}
    \centering
        \caption[Short title for the List of Tables.]{Long caption for this table which is composed by sub-table 1 and sub-table 2.}
        \begin{subtable}{.5\textwidth}
            \centering
                \caption{Sub-table 1.}
            	\begin{tabular}{llr}
            		\toprule
            		id & method & result\\
            		\midrule
            		1 & A & 0.9\\
            		2 & B & 0.8\\
            		\bottomrule
            	\end{tabular}
        \end{subtable}% <---- don't forget this %
        \begin{subtable}{.5\textwidth}
            \centering
                \caption{Sub-table 2.}
            	\begin{tabular}{llr}
            		\toprule
            		id & method & result\\
            		\midrule
            		1 & C & 90\%\\
            		2 & D & 80\%\\
            		\bottomrule
            	\end{tabular}
        \end{subtable}
\label{tab:ex}
\end{table}


\begin{figure}
\centering
 \centering
 \begin{tikzpicture}[scale=0.5]
  \node[draw] at (3,7) {Generative};

  \draw[draw=red,fill=red] (1,1) circle (0.2);
  \draw[draw=red,fill=red] ( 2 , 3.5 ) circle (0.2);
  \draw[draw=red,fill=red] ( 0 , 3 ) circle (0.2);
  \draw[draw=red,fill=red] ( 1 , 2 ) circle (0.2);
  \draw[draw=red,fill=red] ( 2 , 3 ) circle (0.2);
  \draw[draw=red,fill=red] ( 3 , 1 ) circle (0.2);
  \draw[draw=red,fill=red] ( 1 , 4 ) circle (0.2);
  \draw[draw=red,fill=red] (1.8,2.5) circle (0.2);
  %
  \draw[rotate around={35:(2,2)},fill=red, opacity=0.2] (1.8,2.5) ellipse (1.5 and 2.5);
  \draw[rotate around={35:(2,2)},fill=red, opacity=0.21] (1.8,2.5) ellipse (1.2 and 2.2);
  \draw[rotate around={35:(2,2)},fill=red, opacity=0.22] (1.8,2.5) ellipse (0.9 and 1.9);
  %%%%%%%%%%%%%%%%%%%%%
  \draw[draw=cyan,fill=cyan] ( 4 , 3 ) circle (0.2);
  \draw[draw=cyan,fill=cyan] ( 4 , 4 ) circle (0.2);
  \draw[draw=cyan,fill=cyan] ( 5 , 5 ) circle (0.2);
  \draw[draw=cyan,fill=cyan] ( 5 , 3 ) circle (0.2);
  \draw[draw=cyan,fill=cyan] ( 3 , 4 ) circle (0.2);
  \draw[draw=cyan,fill=cyan] ( 3 , 5 ) circle (0.2);
  \draw[draw=cyan,fill=cyan] ( 5 , 3 ) circle (0.2);
  \draw[rotate around={35:(4.5,4)},fill=cyan, opacity=0.2] (4.3,4.2) ellipse (1.5 and 1.8);
  \draw[rotate around={35:(4.5,4)},fill=cyan, opacity=0.21] (4.3,4.2) ellipse (1.2 and 1.5);
  \draw[rotate around={35:(4.5,4)},fill=cyan, opacity=0.22] (4.3,4.2) ellipse (0.9 and 1.2);
  %%%%%%%%%%%%%%%%%%%%%%%%%%%%%%%%%%%%%%%%%
  \node[draw] at (13,7) {Discriminative};
  \draw[ultra thick] (15,0) -- (11,6cm);
  \draw[draw=red,fill=red] ( 11 , 1 ) circle (0.2);
  \draw[draw=red,fill=red] ( 12, 3.5) circle (0.2);
  \draw[draw=red,fill=red] ( 10 , 3 ) circle (0.2);
  \draw[draw=red,fill=red] ( 11 , 2 ) circle (0.2);
  \draw[draw=red,fill=red] ( 12 , 3 ) circle (0.2);
  \draw[draw=red,fill=red] ( 13 , 1 ) circle (0.2);
  \draw[draw=red,fill=red] ( 11 , 4 ) circle (0.2);
  \draw[draw=red,fill=red] (11.8,2.5) circle (0.2);
  %%%%%%%%%%%%%%%%%%%%%
  \draw[draw=cyan,fill=cyan] ( 14 , 3 ) circle (0.2);
  \draw[draw=cyan,fill=cyan] ( 14 , 4 ) circle (0.2);
  \draw[draw=cyan,fill=cyan] ( 15 , 5 ) circle (0.2);
  \draw[draw=cyan,fill=cyan] ( 15 , 3 ) circle (0.2);
  \draw[draw=cyan,fill=cyan] ( 13 , 4 ) circle (0.2);
  \draw[draw=cyan,fill=cyan] ( 13 , 5 ) circle (0.2);
  \draw[draw=cyan,fill=cyan] ( 15 , 3 ) circle (0.2);
 \end{tikzpicture}
\caption{Example of a figure using the package \textit{tikz}}
\label{fig:ex}
\end{figure}
   % Introduction 
\cleardoublepage
\chapter{First real chapter, e.g.: Theoretical Background / Fundamentals}
   % (\chapter{})
\cleardoublepage
\chapter{Second real chapter, e.g.: Methodology}
   % (\chapter{})
\cleardoublepage
%% ... more chapters ....
%\include{conlusion}   % Conclusion (\chapter{Conclusion}  TEXT)
%\cleardoublepage

\appendix
\cleardoublepage

   % appendix A
\cleardoublepage
%\include{thesis10}   % appendix B
%\cleardoublepage
%\include{thesis11}   % appendix C
%\cleardoublepage

%% Do not change, auto-generated lists of figures, tables and literature %%
% Glossar
\printunsrtglossary[type=abbreviations]
\cleardoublepage

% List of figures
\addcontentsline{toc}{chapter}{\listfigurename}
\listoffigures
\cleardoublepage

% List of tables
\addcontentsline{toc}{chapter}{\listtablename}
\listoftables
\cleardoublepage

% Literature list
% %CONFIG:
%\selectlanguage{german}{\addcontentsline{toc}{chapter}{\bibname}}
\selectlanguage{english}{\addcontentsline{toc}{chapter}{\bibname}}

\printbibliography
\end{document}
