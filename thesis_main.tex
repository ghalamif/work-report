% Document class for LME theses: lmedoc
% %LANGUAGE
% %CONFIG
%    The option "german" uses german.sty
%    For english papers, use the "english" option
% Possible types of theses:
% bt - Bachelor's thesis
% mt - Master's thesis
% diss - Dissertation
% sa - Student thesis
% fp - Forschungspraktikum
% pa - Projektarbeit
%\documentclass[german,bt]{lmedoc/lmedoc}
\documentclass[english,fp]{lmedoc/lmedoc}


%%%%%%%%%%%%%%%%%%
% pdflatex and lualatex are supported
% ++ "Umlaut" support
%    The package "inputenc" can be used to write Umlaute or the german double s
%    directly. You need to use the correct encoding, e.g. latin1.
\usepackage{iftex}
\ifPDFTeX
  \usepackage[utf8]{inputenc}
  \usepackage[T1]{fontenc}
  \usepackage{lmodern}
\else
  \ifXeTeX
     \usepackage{fontspec}
  \else 
     \usepackage{luatextra}
  \fi
\fi

%Load additional packages
\usepackage{lmedoc/packages}
% put in this file your abbreviations
% add your abbreviations here
\newabbreviation{svm}{SVM}{Support Vector Machine}

%%%%%%%%%%%%%%%%%%%%%%%%%%%%

% supress messages because of underful hbox, this is not a problem
\hbadness=10000



% some useful commands
\makeatletter % let's define a single dot
\DeclareRobustCommand\onedot{\futurelet\@let@token\@onedot}
\newcommand{\@onedot}{\ifx\@let@token.\else.\null\fi\xspace}
\makeatother



% Sets the bib file

\addbibresource{literature.bib}

% When writing a large document, it is sometimes useful to work on selected sections of the document.
% Use this command to only build the document partially. Speeds up the developement cycle.
% For the final product, this has to be commented out.
%\includeonly{introduction,appendix,foo,bar}


\pagenumbering{roman}

\begin{document}
\clearpage
% %CONFIG
% This is for students' theses
  \begin{deckblatt}
    \Titel{Titel} % Title
    \Name{Name} % Last name
    \Vorname{Vorname} % Given name
    \Geburtsort{Geburtsort} % Place of birth
    \Geburtsdatum{Geburtsdatum} % Date of birth
    \Betreuer{Betreuer} % Advisor
    \Start{Start} % Start of thesis
    \Ende{Ende} % End of thesis
    %\ZweitInstitut{ZweitInstitut} % Cooperation partner
  \end{deckblatt}


\cleardoublepage
% Only necessary for thesis

Ich versichere, dass ich die Arbeit ohne fremde Hilfe und ohne Benutzung
anderer als der angegebenen Quellen angefertigt habe und dass die Arbeit
in gleicher oder "ahnlicher Form noch keiner anderen Pr"ufungsbeh"orde
vorgelegen hat und von dieser als Teil einer Pr"ufungsleistung
angenommen wurde. Alle Ausf"uhrungen, die w"ortlich oder sinngem"a"s
"ubernommen wurden, sind als solche gekennzeichnet.
\\


\cleardoublepage


\begin{center}
\bfseries
% Abstract in German
{\selectlanguage{german}"Ubersicht}
\normalfont
\end{center}


\vspace{5.0cm}

\begin{center}
\bfseries
% Abstract in English
{\selectlanguage{english}Abstract}
\normalfont
\end{center}


\cleardoublepage

\tableofcontents

\cleardoublepage \pagenumbering{arabic}

\include{introduction}   % Introduction 
\cleardoublepage
\include{background}   % (\chapter{})
\cleardoublepage
\include{methodology}   % (\chapter{})
\cleardoublepage
%% ... more chapters ....
%\include{conlusion}   % Conclusion (\chapter{Conclusion}  TEXT)
%\cleardoublepage

\appendix
\cleardoublepage
\include{appendix}   % appendix A
\cleardoublepage
%\include{thesis10}   % appendix B
%\cleardoublepage
%\include{thesis11}   % appendix C
%\cleardoublepage

%% Do not change, auto-generated lists of figures, tables and literature %%
% Glossar
\printunsrtglossary[type=abbreviations]
\cleardoublepage

% List of figures
\addcontentsline{toc}{chapter}{\listfigurename}
\listoffigures
\cleardoublepage

% List of tables
\addcontentsline{toc}{chapter}{\listtablename}
\listoftables
\cleardoublepage

% Literature list
% %CONFIG:
%\selectlanguage{german}{\addcontentsline{toc}{chapter}{\bibname}}
\selectlanguage{english}{\addcontentsline{toc}{chapter}{\bibname}}

\printbibliography
\end{document}
